\chapwithtoc{Úvod}
\label{chap:intro}
%\todo[inline]{Sem by som asi nedavala ziadne section}
Problém anonymizácie dát je dôležitý v rôznych oblastiach, napríklad v oblasti verejnej správy či
v oblasti marketingu. Pod pojmom anonymizácia dokumentov si môžeme predstaviť vymazanie či skrytie údajov 
alebo iných citlivých informácií. Možností, ako pristupovať k anonymizácii dokumentov, je veľa. 
\newline

Typickým miestom, kde sa stretávame s anonymizáciou dokumentov, je oblasť verejnej správy.
V Českej republike majú organizácie verejnej správy povinnosť zverejňovať informácie o svojej činnosti,
k čomu patrí aj zverejňovanie uzavretých zmlúv nad určitú čiastku do \textit{registra zmlúv}, 
ktorý je verejne prístupný. Nachádzajú sa tu nielen informácie o predmete zmlúv, zmluvných stranách a cene,
ale takisto všetky súbory, ktoré sú súčasťou zmlúv. Register zmlúv je významným nástrojom, ktorý zlepšuje 
transparentnosť; podstatou je kontrolovať a mať možnosť obmedziť korupciu a zneužívanie verejnej moci 
kvôli uzatváraniu nevýhodných zmlúv.
\newline

Aj napriek tomu, že zverejňovanie dát v registri zmlúv je právne vynútiteľné, nezabezpečuje to automaticky
možnosť jednoduchého vyhľadávania či analýzy týchto dát. K tomu bol vytvorený projekt, webový portál 
\textit{Hlídač smluv}, ktorý má za úlohu zlepšiť prístup k registru zmlúv. Neskôr, po skombinovaní ďalších 
verejne prístupných dát z registrov a databáz, sa vytvoril projekt Hlídač státu \cite{HlidacStatu}, ktorý má za úlohu zlepšiť prístup k verejným informáciám. Poskytuje
napríklad plnohodnotné vyhľadávanie v texte zmlúv.
\newline


\begin{hyphenrules}{nohyphenation}
V registri zmlúv sú dokumenty z rôznych oblastí, napríklad z oblasti zdravotníctva, školstva, realitných služieb alebo
IT projektov. V prípade, že dokumenty obsahujú citlivé údaje, sú častokrát anonymizované. V súčasnej dobe neexistuje 
štatistický nástroj, ktorý by znázorňoval koľko percent v takýchto dokumentoch je zanonymizovaných.
\end{hyphenrules}

\newpage

\begin{hyphenrules}{nohyphenation}
Cieľom práce je zaoberať sa anonymizovanými PDF dokumentmi a vytvorenie nástroja, 
ktorý bude schopný detegovať anonymizované časti dokumentu, využiť grafické metódy používané pri počítačovom videní a ďalších algoritmov na spracovanie obrazu a navrhnúť tak systém, ktorý umožní na základe dostupných dát vyhodnotiť percento anonymizácie jednotlivých zmlúv pri použití konkrétnych implementačných metód a následne nasadiť túto implementáciu na webový portál Hlídače státu.
\newline

Hlavným prínosom práce je vytvorenie systému na porovnávanie jednotlivých odvetví, ktoré zverejňujú zmluvy vzhľadom na percento anonymizácie a
tvorba štatistiky vzhľadom na anonymizáciu dokumentov relatívne k jednotlivým oblastiam.
\newline

Práca je štruktúrovaná nasledovne: \ref{chap:FirstChapter}. kapitola rieši definíciu anonymizácie, právne aspekty a dôvody pre anonymizáciu a možné metódy a techniky, ktorými sa dokumenty anonymizujú. V \ref{chap:SecondChapter}. kapitole je špecifikovaný konkrétny problém a~zadanie, ktorému sa v práci venujeme. \ref{chap:ThirdChapter}. kapitola je venovaná popisu vstupných dát, ich obsah a štruktúra. Takisto je tu popísaný proces získavania dát a ich príprava na ďalšie spracovanie a extrakcia relevantných informácií. V \ref{chap:FourthChapter}. kapitole je popísaný proces detekcie anonymizovaných častí dokumentov. V kapitole \ref{chap:FifthChapter} je popísaný algoritmus, ktorý v systéme používame.
\Cref{chap:SixthChapter} je venovaná implementácii riešenia, použité technológie a nástroje, architektúra a dizajn systému, prípadové štúdie a ďalšie detaily implementácie. V závere kapitoly \ref{chap:conclusion} zhŕňame mimo dosiahnutých výsledkov aj osobné zistenia a odporúčania pre ďalší výskum.  
\end{hyphenrules}