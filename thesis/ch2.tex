\chapter{Špecifikácia problému}
\label{chap:SecondChapter}
\section{Identifikácia kľúčových výziev}
\subsection{Komplexita anonymizovaných dokumentov}
\begin{hyphenrules}{nohyphenation}
Keďže zákon \cite{ZakonyProLidi2015-340} presne neukladá, v akom formáte majú byť zmluvy zverejňované a~neexistuje ani právna úprava, ktorá by regulovala spôsoby anonymizácie, v~registri zmlúv preto nájdeme mnoho rôznych foriem a typov dokumentov.
Základné rozdelenie zverejnených zmlúv je, či sú zmluvy digitálne, t. j. či sú originály zmlúv v digitálnej podobe, alebo sú zmluvy skenované z fyzických originálov (najčastejšie sken A4 dokumentov). 
\newline 

Z pohľadu detekcie anonymizovaných častí dokumentov je vhodnejšie analyzovať digitálne zmluvy, pretože neobsahujú artefakty spôsobené skenovaním a z dôvodu neprítomnosti šumu je preto jednoduchšie analyzovať takéto dokumenty.
Pri preskenovaných fyzických origináloch častokrát dochádza k nedokonalému skenu, kedy sa pri skenovaní dokumentu nedôkladne preskenuje daný dokument. Konkrétnymi príkladmi takýchto artefaktov môžu byť rohy papierov, kde sú viacstranové zmluvy zospinkované, a teda nedôjde k 
dôkladnému priloženiu skenovanej predlohy na plochu skenera, alebo sa pri skenovaní stratí informácia o farbe (ak je daný dokument skenovaný do čiernobielej), čo môže spôsobiť problémy pri hľadaní začiernených plôch. Zväčša sa jedná o úradné pečiatky, logá firiem, obrázkové prílohy či záhlavia tabuliek.


\subsection{Rozpoznanie anonymizovaných oblastí}
Vzhľadom na vyššie spomínané spôsoby anonymizácie je náročné kategorizovať jednotlivé typy a na základe toho určiť pomer anonymizovaných oblastí vzhľadom na obsah dokumentu. Pri rozpoznávaní daných oblastí je dôležité mať na pamäti túto rôznorodosť a adekvátne navrhnúť algoritmus tak, aby dokázal detegovať čo najviac techník a čo najpresnejšie určiť ich rozsah. Keďže sa v práci zameriavame na algoritmické riešenie bez použitia umelej inteligencie a machine learning, spôsob na rozpoznanie týchto oblastí je určený prevažne implementáciou rôznych algoritmov počítačového videnia, ktoré sú bližšie popísané v \ref{chap:FourthChapter}. kapitole.

\section{Prehľad existujúcich riešení a ich obmedzenia}
\subsection{Súčasné metódy a nástroje}
Vzhľadom na špecifickosť problému a rozsah uplatnenia v súčasnej dobe neexistujú plnohodnotné verejne dostupné nástroje, ktoré by boli zamerané na detekciu anonymizovaných oblastí v zmluvách. Každopádne s technológiami ako machine learning a AI je pravdepodobné, že v blízkej dobe vzniknú nástroje využívajúce práve tieto metódy na riešenie tohto typu problému.

\section{Definovanie požiadaviek na riešenie}

Žiadanými parametrami sú správnosť detekcie oblasti, kde došlo k anonymizácii, presnosť detekcie anonymizovaných oblastí a presný odhad anonymizovanej oblasti vzhľadom na obsah dokumentu. Správnosť detekcie oblasti zaručuje, že nedôjde k označeniu miesta na dokumente ako anonymizované, keď v skutočnosti nie je, napríklad záhlavie tabuľky, obrázok, zvýraznený text, pečiatka či logo. Pod~presnosťou rozumieme čo najpresnejšie ohraničenie anonymizácie. Napríklad, ak sa jedná o prelepenie nálepkou alebo ak oblasť anonymizácie hraničí s textom, chceme, aby sme ohraničili oblasť presne celú bez čo najmenšej aproximácie. Presný odhad anonymizovanej oblasti je požiadavka vzťahujúca sa na možný odhad toho, koľko daná oblasť zakrýva. V niektorých prípadoch, kedy je zamazaný celý riadok alebo nejaká časť riadku, vieme odhadnúť, koľko znakov, respektíve percent vzhľadom na celý text je prekrytých. Ak je prekrytých viac riadkov, je zložitejšie určiť percento prekrytia vzhľadom na to, že text nemusí byť formálne rozložený (napr. koniec odseku, vynechaný riadok alebo viac riadkov). Čím viac plochy je prekrytej, tým ťažšie je odhadnúť, koľko údajov bolo anonymizovaných.
\newline

Z užívateľského hľadiska je požadovaná okrem spomínaných požiadaviek aj jednoduchosť používania softvéru a rýchlosť analýzy daného dokumentu, resp. dokumentov. Požadujeme, aby systém dokázal spracúvať väčšie množstvo dokumentov v rýchlom čase a v prípade chyby (problém pri sťahovaní či otváraní dokumentu a pod.) pokračoval bez obmedzení a o chybe uživateľa~informoval. 
\end{hyphenrules}
