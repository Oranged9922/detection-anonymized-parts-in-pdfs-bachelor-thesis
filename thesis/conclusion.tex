\label{chap:conclusion}
\chapter{Záver}
\section{Zhrnutie dosiahnutých výsledkov}
\begin{hyphenrules}{nohyphenation}
V práci sme sa zamerali na vývoj a testovanie programu s algoritmom na detekciu anonymizovaných oblastí v dokumentoch. Výsledky, ktoré sme dosiahli, zahŕňajú:
\begin{itemize}
    \item Vývoj algoritmu na detekciu anonymizovaných oblastí

    Implementovali sme algoritmus schopný detegovať anonymizované oblasti rôznych typov v skenovaných a digitálnych dokumentoch. Algoritmus sme navrhli tak, aby bol schopný detegovať rôzne anonymizované oblasti, vrátane nepravidelných a farebných či šumových anonymizácií.
    \item Testovanie a validácia

    Algoritmus sme testovali na mnohých dokumentoch, ktoré zahŕňali rôzne typy anonymizácie. Výsledky ukázali vysokú presnosť detekcie v prípade bežných typov anonymizovania, ako sú čierne obdĺžniky a farebné nálepky.
    \item Špecifické výzvy

    Podrobnou analýzou sme identifikovali problémy spojené s detekciou anonymizovaných oblastí v prípade nekvalitných skenov a dokumentov s~rôznymi druhmi rušivých elementov. Tieto výzvy boli čiastočne vyriešené, no stále je tu priestor na ďalší výskum a zlepšenie.

    \item Možnosti integrácie a nasadenia

    V súčasnej dobe nie je aplikácia integrovaná so žiadnym iným systémom ani nasadená ako samostatná aplikácia. Aplikácia je však vhodná k použitiu ako referenčná implementácia. V budúcnosti je možnosť integrácie tejto aplikácie v rámci Hlídače státu\cite{HlidacStatu}.
\newline 

\end{itemize}

\section{Doporučenia pre ďalší výskum}
Vychádzajúc z našich zistení a výsledkov máme niekoľko doporučení pre budúci výskum v oblasti detekcie anonymizovaných oblastí:
\begin{itemize}
    \item Vyššia odolnosť voči rušivým elementom

    Významným problémom, ktorému je možné venovať sa bližšie, je redukcia rušivých elementov v dokumentoch, obzvlášť v prípade nekvalitných skenov, ale aj digitálných dokumentov. Jedná sa o logá firiem, úradné pečiatky, záhlavia tabuliek či iné rušivé elementy, ktoré sú problémom pre náš algoritmus pri správnom vyhodnocovaní a vyriešením tohoto problému by sa prispelo k redukcii falošne pozitívnych detekcií.

    \item Využitie metód strojového učenia

    Implementácia konvolučných sietí alebo iných metód strojového učenia môže zvýšiť presnosť a správnosť detekcie. Problémom je tu však absencia datasetu, na základe ktorého by mohla byť takáto konvolučná sieť namodelovaná.

\end{itemize}

\section{Osobné zistenia a závery}
Pri práci na tejto téme sme si uvedomili viacero kľúčových bodov, ktoré sú dôležité pre správne pochopenie z pohľadu technického vývoja. Medzi najdôležitejšie body uvádzame komplexnosť a rôznorodosť reálnych dát. Rôzne formáty dokumentov, kvalita skenov, typy anonymizácií predstavovali a predstavujú výzvy, ktoré si vyžadujú komplexné a flexibilné riešenia. 
\newline

Ďalším bodom, ktorý spomenieme, je spoľahlivosť a presnosť. Falošne pozitívne výsledky môžu mať závažné dopady, obzvlášť v prípade, že by sa na základe týchto výsledkov hodnotila transparentnosť či dôvernosť inštitúcií, ktoré takéto dokumenty, resp. zmluvy zverejňujú.
\newline

Dúfame, že tieto zistenia a závery poslúžia ako základ pre ďalší vývoj a výskum v oblasti detekcie anonymizovaných oblastí v dokumentoch.
\end{hyphenrules}