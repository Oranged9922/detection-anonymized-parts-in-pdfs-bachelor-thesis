\chapwithtoc{Úvod}

Problém anonymizácie dát je dôležitý v rôznych oblastiach, napríklad v oblasti verejnej správy či
v oblasti marketingu. Pod pojmom anonymizácie dokumentov si môžeme predstaviť vymazanie či skrytie údajov 
alebo iných citlivých informácií. Možností, ako pristupovať k anonymizácii dokumentov, je veľa. 

Typickým miestom, kde sa stretávame s anonymizáciou dokumentov, je oblasť verejnej správy.
V Českej republike majú organizácie verejnej správy povinnosť zverejňovať informácie o svojej činnosti,
k čomu patrí aj zverejňovanie uzavrených zmlúv nad určitú čiastku do \textit{registru zmlúv}, 
ktorý je verejne prístupný. Nachádzajú sa tu nielen informácie o predmete zmlúv, zmluvných stranách a cene,
ale takisto všetky súbory, ktoré sú súčasťou zmlúv. Register zmlúv je výnzamným nástrojom, ktorý zlepšuje 
transparentnosť; podstatou je kontrolovať a mať možnosť obmedziť korupciu a zneužívanie verejnej moci 
kvôli uzatváraniu nevýhodných zmlúv.

Aj napriek tomu, že zverejňovanie dát do registra zmlúv je právne vynútiteľné, nezabezpečuje to automaticky
možnosť jednoduchého vyhľadávania či analýzy týchto dát. K tomu bol vytvorený projekt, webový portál 
\textit{Hlídač smluv}, ktorý má za úlohu zlepšiť prístup k registru zmlúv. Neskôr, po skombinovaní ďalších 
verejne prístupných dát z registrov a databází, sa vytvoril projekt \textit{Hlídač státu}
\footnote{\url{https://hlidacstatu.cz}}, ktorý má za úlohu zlepšiť prístup k verejným informáciám. Poskytuje
napríklad plnohodnotné vyhľadávanie v texte zmlúv.

V registri zmlúv sú dokumenty z rôznych oblastí, napríklad z oblasti zdravotníctva, školstva, realitných služieb alebo
IT projektov. V prípade, že dokumenty obsahujú citlivé údaje, sú častokrát anonymizované. V súčasnej dobe neexistuje 
štatistický nástroj, ktorý by znázorňoval koľko percent v takýchto dokumentoch je zanonymizovaných.

V tejto práci sa budeme zaoberať anonymizovanými PDF dokumentmi a budeme sa snažiť vytvoriť nástroj, 
ktorý bude schopný detekovať anonymizované časti dokumentu, využiť metódy strojového učenia a ďalších algoritmov na 
spracovanie obrazu a navrhnúť tak systém, ktorý umožní na základe dostupných dát vyhodnotiť percento anonymizácie 
jednotlivých zmlúv pri použití konkrétnych implementačných metód.

Túto implementáciu potom bude možné nasadiť na webový portál Hlídače státu.

Pri implementácii je nutné uvedomiť si rozličné spôsoby anonymizovania týchto dát, z ktorých najčastejšími sú:
\begin{itemize}
    \item prekrytie časti dokumentu čiernou farbou
    \item prekrytie časti dokumentu bielou farbou
    \item zašumenie časti dokumentu
\end{itemize}


% Introduction should answer the following questions, ideally in this order:
% \begin{enumerate}
% \item What is the nature of the problem the thesis is addressing?
% \item What is the common approach for solving that problem now?
% \item How this thesis approaches the problem?
% \item What are the results? Did something improve?
% \item What can the reader expect in the individual chapters of the thesis?
% \end{enumerate}

% Expected length of the introduction is between 1--4 pages. Longer introductions may require sub-sectioning with appropriate headings --- use \texttt{\textbackslash{}section*} to avoid numbering (with section names like `Motivation' and `Related work'), but try to avoid lengthy discussion of anything specific. Any ``real science'' (definitions, theorems, methods, data) should go into other chapters.
% \todo{You may notice that this paragraph briefly shows different ``types'' of `quotes' in TeX, and the usage difference between a hyphen (-), en-dash (--) and em-dash (---).}

% It is very advisable to skim through a book about scientific English writing before starting the thesis. I can recommend `\citetitle{glasman2010science}' by \citet{glasman2010science}.
